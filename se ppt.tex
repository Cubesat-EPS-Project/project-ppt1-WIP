\documentclass[aspectratio=169]{beamer}
\hypersetup{pdfpagemode=FullScreen}
\usepackage[utf8]{inputenc}
\usepackage[T1]{fontenc}
\usepackage{lmodern}
\usepackage[english]{babel}
\usepackage{graphicx}
\usepackage{amsmath}
\usepackage{amsfonts}
\usepackage{amssymb}
%\usetheme{CambridgeUS}
%\usetheme{Berlin}
%\usetheme{Montpellier}
\usetheme{Boadilla}
\useinnertheme{circles}
\renewcommand{\footnotesize}{\fontsize{6pt}{10pt}\selectfont}
%\usepackage[nottoc]{tocbibind}
\usepackage{cite}
%\useoutertheme{miniframes} % Alternatively: miniframes, infolines, split
%\useinnertheme{circles}
%\usecolortheme{default}
%\usecolortheme{dolphin}
\usepackage{hyperref}
\usefonttheme[onlymath]{serif}
\setbeamertemplate{caption}[numbered]
\setbeamertemplate{bibliography item}{\insertbiblabel}
%\hypersetup{
%	colorlinks=true,
%	linkcolor=blue,
%	filecolor=magenta,      
%	urlcolor=cyan,
%}
\renewcommand{\figurename}{Fig.}
\setbeamertemplate{section in toc}[sections numbered]
\insertsectionnavigationhorizontal{.5\textwidth}{\hskip0pt plus1filll}{}
%\setbeamertemplate{frametitle}[rounded]
\setbeamertemplate{blocks}[rounded][shadow]



\defbeamertemplate*{footline}{CambridgeUS theme}
{
	\leavevmode%
	\hbox{%
		\begin{beamercolorbox}[wd=.333333\paperwidth,ht=2.25ex,dp=1ex,center]{author in head/foot}%
			\usebeamerfont{author in head/foot}%\insertshortauthor
			~~\insertshortinstitute
		\end{beamercolorbox}%
		\begin{beamercolorbox}[wd=.333333\paperwidth,ht=2.25ex,dp=1ex,center]{title in head/foot}%
			\usebeamerfont{title in head/foot}\insertshorttitle
		\end{beamercolorbox}%
		\begin{beamercolorbox}[wd=.333333\paperwidth,ht=2.25ex,dp=1ex,right]{date in head/foot}%
			\usebeamerfont{date in head/foot}\hspace*{2em}
			Slide: \insertframenumber{} of \inserttotalframenumber\hspace*{2ex}
	\end{beamercolorbox}}%
	\vskip0pt%
}



\defbeamertemplate*{headline}{CambridgeUS theme}
{
	\leavevmode%
	\hbox{%
		\begin{beamercolorbox}[wd=.333333\paperwidth,ht=2.25ex,dp=1ex,center]{author in head/foot}%
			\usebeamerfont{author in head/foot}%\insertshortauthor
			~~%\insertshortinstitute
		\end{beamercolorbox}%
		\begin{beamercolorbox}[wd=.333333\paperwidth,ht=2.25ex,dp=1ex,center]{title in head/foot}%
			\usebeamerfont{title in head/foot}%\insertshorttitle
		\end{beamercolorbox}%
		\begin{beamercolorbox}[wd=.333333\paperwidth,ht=2.25ex,dp=1ex,right]{date in head/foot}%
			\usebeamerfont{date in head/foot}\insertshortdate{}\hspace*{2em}
			\textsl{\emph{{{{\tiny {\tiny }}}}}}%\insertframenumber{} of \inserttotalframenumber\hspace*{2ex}
	\end{beamercolorbox}}%
	\vskip0pt%
}

  
  
  \definecolor{UniBlue}{RGB}{83,121,170}
  
  \setbeamercolor{title}{fg=white, bg=UniBlue}
  %\setbeamercolor{title in head/foot}{fg=black,bg=lightgray}
  %\setbeamercolor{author in head/foot}{fg=white,bg=black}
 %\setbeamercolor{frametitle}{fg=white,bg=black}
  %\setbeamercolor{block title}{fg=white,bg=black}
  %\setbeamercolor{block body}{bg=lightgray, fg=black}
  %\setbeamercolor{date in head/foot}{fg=white, bg=black}
  %\setbeamercolor{item}{fg=gray}
  




  
\begin{document}
	%\author{}
	\title{DESIGN AND IMPLEMENTATION OF
		CUBESAT ELECTRICAL POWER
		SYSTEM}
	
	\subtitle{}
	%\logo{}
     \author{Presented by :\\Naveen A.B. | TRV19EE038}
      \author{Presented by :\\Naveen A.B. | TRV19EE038}
	\institute{GEC Barton Hill,Thiruvananthapuram}
	\date{\today}
	
	
	\setbeamercovered{transparent}
\setbeamertemplate{navigation symbols}
{%
	\hbox{%
		\hbox{\insertslidenavigationsymbol}
		\hbox{\insertframenavigationsymbol}
		\hbox{\insertsubsectionnavigationsymbol}
		\hbox{\insertsectionnavigationsymbol}
		\hbox{\insertdocnavigationsymbol}
		\hbox{\insertbackfindforwardnavigationsymbol}}%
}
	\begin{frame}[plain]
	\maketitle
	\center{{Guided by: Prof. Simi Raj}}
%	\center{{SEMINAR PRESENTATION}}
\end{frame}


\begin{frame}
\frametitle{Contents}




\tableofcontents
\end{frame}


\section{Introduction}
\begin{frame}{Introduction}
	
	\begin{itemize}
		\item Form factor of
		of 10 cm cube
		\item  Cost-effective, timely and relatively easy to accomplish
		\item  The Electrical Power System (EPS) is an electronic circuit board that is designed supply and manage energy to the Cubesat subsystems.


		
	\end{itemize} 
	
\end{frame}


\section{Objective}
\begin{frame}
	
\frametitle{Objective}
 To design and implement a power generation, storage and distribution system for a CubeSat 
 
 

\end{frame}




\section{Dynamic/Rheostatic Braking}

\begin{frame}
	\frametitle{Dynamic/Rheostatic Braking}
	\begin{itemize}
		\item Works on the principle of reversibility of electrical machines
		\item Traction motors are switched into generator mode
		\item Power from braking is dissipated as heat in brake grid choppers or
		resistors
		\item Thus, the kinetic energy of the rolling stock is converted to heat energy

		
	\end{itemize}
\end{frame}


\begin{frame}
	\frametitle{Dynamic/Rheostatic Braking (Contd.)}
	Pros:
	\begin{itemize}
		\item Reduction of wear and tear of brake shoes and wheels
		\item Lesser chance of brake fade
		\item Speed control on downgrades
		\item Very short response time
		
	\end{itemize}
Cons:
	\begin{itemize}
		\item Wasted power is about 10–30\% of the total
		locomotive energy usage \cite{p4}
	\item Large cooling fans necessary for thermal protection
	\item Excessive heat may damage or disable the resistors
	\item Ineffective at very low speeds
	
\end{itemize}
\end{frame}


\section{Commonly used Energy Storage Mechanisms}
\begin{frame}{Commonly used Energy Storage Mechanisms}
	
	
	\begin{enumerate}
		\item Electric: Regenerated energy is stored in electrical storage devices
		such as batteries and super capacitors
		\item Hydraulic/Pneumatic: Energy is converted into internal energy of a
		liquid or compressed gas or a vacuum
		\item Mechanical: Energy is stored in the form of mechanical energy of
		rotation or translational motion
	\end{enumerate}
	
	
\end{frame}

\section{HESS Architecture}
\begin{frame}{HESS Architecture}
		\begin{figure}
		\includegraphics[width=12.999cm]{hess.png}
	\caption{HESS Architecture (Source: Ref. \cite{p1})}
		\label{fig:hess}
		
	\end{figure}
\end{frame}
	
	
	
	

\begin{frame}{HESS Architecture: Power Generation System}
	\begin{minipage}{0.5\textwidth}
		\begin{itemize}
			
			\item Main (MG) and Auxiliary generator (AG) coupled to diesel engine shaft

			\item MG output is rectified for powering DC Traction motors
			\item AG produces a 3-phase AC output
			\item AG output powers compressors, blowers, sanding systems, etc. (20\% FL)
		\end{itemize} 
	\end{minipage}
	\begin{minipage}{0.3\textwidth}
		\begin{figure}[h]
			\centering
			\includegraphics[width=6.7cm]{powgen.png}
			
			
		
			\label{fig:pow1}
			
		\end{figure}
	\end{minipage}
\end{frame}



\begin{frame}{HESS Architecture: Traction and Braking System}
	\begin{minipage}{0.5\textwidth}
		
		\begin{itemize}
			
			\item DC motor is used for traction
			\item $R_{DB}$ - Dynamic Brake grid resistance
			\item $S_{T}$ - Traction Motor switch
			\item $S_{DB}$ - Dynamic Brake switch
			\item $S_{RB}$ - Regenerative Brake switch
		\end{itemize} 
	\end{minipage}
	\begin{minipage}{0.3\textwidth}
		
		\begin{figure}
			\includegraphics[width=6.5cm]{tracn.png}
			\begin{center}
			\caption{Traction and Braking diagram}
		\end{center}
			\label{fig:trac2}
			
		\end{figure}
	
	\end{minipage}
\end{frame}




\begin{frame}{HESS Architecture: Energy Processing}

		\begin{itemize}
			
			\item Li-ion batteries and super capacitors act as energy storage
			\item SCs take away all peak loads from the battery

		\end{itemize} 


		\begin{figure}
			\includegraphics[width=8.5cm]{pewer.png}
			\caption{Energy Storage and Conversion\footnote{IEEE Transactions on Industrial Electronics, VOL. 68, NO. 10, OCTOBER 2021; pp. 9083
					\cite{p1}	}}
			\label{fig:ss42}
			
		\end{figure}

\end{frame}

\begin{frame}{HESS Architecture: Energy Processing (Contd.)}
	
	\begin{itemize}
		
		\item Two DAB converters connected in input series output parallel fashion
		\item The DABs feeds regenerated energy into a common DC bus
		
		
		
		
	\end{itemize} 
	\begin{minipage}{0.5\textwidth}
		\begin{center}
	
	\begin{figure}
	\includegraphics[width=6.8cm]{dab.png}
	\begin{center}
		\caption{Dual Active Bridge (Source: Ref. \cite{p5})}
	\end{center}
	\label{fig:dab}
	
\end{figure}
\end{center}
\end{minipage}
\begin{minipage}{0.4\textwidth}
	
	
	\begin{figure}
		\includegraphics[width=4.4cm]{isop.png}
		\begin{center}
			\caption{ISOP configuration (Source: Ref. \cite{p5})}
		\end{center}
		\label{fig:isop}
		
	\end{figure}
	
\end{minipage}
\end{frame}

\begin{frame}{HESS Architecture: Energy Processing Contd.)}
	
	\begin{itemize}
		
		\item Bidirectional non-isolated dc–dc converters are used to charge and discharge the batteries
		\item Decoupling between batteries, SCs, and the dc bus allows
		both storage devices to operate at a wider range of SoC\footnote{State of Charge}
		\item Volumetric efficiency of HESS is thus improved
		\item 3-$\Phi$ 2-level VSI with passively damped LCL filter is used to transfer power from DC bus to AC aux loads.
	\end{itemize} 
	

	
\end{frame}


\section{Power Flow}
\begin{frame}{Power Flow}
	
	\begin{itemize}
		\item  Generated power must always be equal
		to the demanded power to ensure the system stability
	\end{itemize} 
\begin{block}{Power Flow Equation}
	\label{Pewer flow}
	On neglecting system losses,
		\begin{equation}
			P_{reg} + P_{gen} - P_{L} \pm P_{HESS} = 0
		\end{equation}
	\begin{itemize}

				\item $P_{reg}$ - Power regenerated from dynamic braking
	\item $P_{gen}$ - Power supplied by the diesel generator
	\item $P_{L}$ -  Power demanded by the locomotive auxiliary loads
	\item $P_{HESS}$ - Power available in the HESS
\end{itemize}	
\end{block}	
	
	
\end{frame}



\begin{frame}
	\frametitle{Power Flow (Contd.)}
	\begin{itemize}
		\item The regenerated
		power is used primarily by the loads during braking, while the
		surplus is stored
		\item If the HESS\footnote{HESS: Hybrid
			Energy Storage System} is fully charged, the control strategy must reduce
		$P_{reg}$ until it matches the load demand
		\item If the HESS reaches its minimum
		SoC, the load is to be fed by the regenerated power
		or the diesel generator
	\end{itemize}
\end{frame}




\section{Modes of Operation}
\begin{frame}{Modes of Operation}
	\label{frame:modes}
	\begin{figure}
		\includegraphics[width=12.999cm]{hess.png}
		\caption{HESS Architecture (Source: Ref.  \cite{p1})}
		\label{fig:hess2}
		
	\end{figure}
\hyperlink{frame:mode1}{1}, \hyperlink{frame:mode2}{2}, \hyperlink{frame:mode3}{3}, \hyperlink{frame:mode4}{4}, \hyperlink{frame:mode5}{5}
\end{frame}

\begin{frame}
	\frametitle{Modes of Operation: Mode 1 - Under-voltage Protection}
	\label{frame:mode1}
	\begin{itemize}
		
		\item When the auxiliary load demand exceeds the discharge
		capacity of the batteries
		\item DAB ISOP converter does not regenerate power to the dc bus (not braking)
		\begin{center}
			

		
	 \begin{equation}{\therefore P_{gen}  + P_{HESS} = P_{L}}\end{equation}

\end{center}
	\end{itemize}
	
\end{frame}



\begin{frame}
	\frametitle{Modes of Operation: Mode 2 - Discharging}
	\label{frame:mode2}
	\begin{itemize}
		\item  \hyperlink{frame:modes}{Entire auxiliary load is supplied only by the HESS or with the aid of regenerated power}
		\item Auxiliary generator provides the reactive power required by the load
 	\begin{equation}{\therefore P_{reg} + P_{HESS} = P_{L}}\end{equation}
	\begin{equation} {Or, P_{HESS} \geq P_{L}} \end{equation}
		
	\end{itemize}
	
\end{frame}




\begin{frame}
	\frametitle{Modes of Operation: Mode 3 - Inverter}
	\label{frame:mode3}
	\begin{itemize}
		\item \hyperlink{frame:modes}{Auxiliary loads absorb all the regenerated
			power}
		\item  HESS	does not operate/charge
		\item The dc bus voltage is regulated by the discharging storage system
		\item A portion of this power is used to recharge
		only the SCs
	\begin{equation}	 {\therefore P_{reg} = P_{L} + P_{SC}} \end{equation}
		
	\begin{equation}	Or,{P_{reg} = P_{L}}  \end{equation}
		
	\end{itemize}
	
\end{frame}

\begin{frame}
	\frametitle{Modes of Operation: Mode 4 - HESS Charging}
	\label{frame:mode4}
	\begin{itemize}
		\item \hyperlink{frame:modes}{The inverter supplies the entire load demand}
		\item  Batteries and SCs are recharged with excess
		regenerated power
		\item HESS regulates the dc bus voltage 
	\begin{equation}  {\therefore P_{reg} = P_{L} + P_{HESS}} \end{equation}
	\end{itemize}
	
\end{frame}

\begin{frame}
	\frametitle{Modes of Operation: Mode 5 - Over-voltage Protection}
	\label{frame:mode5}
	\begin{itemize}
		\item  \hyperlink{frame:modes}{Acts whenever the regenerated power exceeds all system}
		demand
		\item The energy storage system is fully charged and auxiliary loads
		are off
		\item Regeneration is turned off
		\begin{equation}  {When, P_{reg} > P_{L} + P_{HESS}} \end{equation}
	\end{itemize}
	
\end{frame}


\section{Advantages and Disadvantages}
\begin{frame}
	\frametitle{Advantages and Disadvantages}
Advantages:
	\begin{itemize}
		\item  Improved electric braking range and effectiveness
		\item  Lesser load on auxiliary generator
		\item  Non - intrusive system
		\item  HESS can be modified to expand it's capability
	\end{itemize}

Disadvantages:
	\begin{itemize}
		\item Lower lifespan of the batteries 
		\item  Maintenance cost

	\end{itemize}	
	


\end{frame}



\section{Conclusion}
\begin{frame}
	\frametitle{Conclusion}
	\begin{itemize}
		\item  Focus on
		using the recovery energy to supply only auxiliary loads has reduced the cost and size of the system
		\item As the system presented is non-intrusive, it can be used to retrofit the existing systems
		\item There is potential for full hybridization of heavy haul locomotives.
	\end{itemize}
	
\end{frame}






\begin{frame}[allowframebreaks]{References}

\begin{thebibliography}{9}
	
		\bibitem[1]{p1}
Moraes, Caio Guilherme da Silva and Brockveld, Sergio Luis and Heldwein, Marcelo Lobo and Franca, André Stanzani and Vaccari, Anderson Silva and Waltrich, Gierri (2021)
\newblock Power Conversion Technologies for a Hybrid Energy Storage System in Diesel-Electric Locomotives
\newblock \emph{IEEE Transactions on Industrial Electronics}  vol. 68, no. 10, pp. 9081-9091.

	 % Reduce the font size in the bibliography

\bibitem[2]{p2}
Spiryagin, Maksym (2014)
\newblock Ground Vehicle Engineering : Design and Simulation of Rail Vehicles
\newblock \emph{CRC Press}.
	
			\bibitem[3]{p3}
	T. Letrouve, W. Lhomme, J. Pouget, and A. Bouscayrol (2014)
	\newblock Different hybridization rate of a diesel-electric locomotive
	\newblock \emph{IEEE Vehicle Power Propulsion Conference}  pp. 1-6.
	
	
	
			\bibitem[4]{p4}
Y. Sun, C. Cole, M. Spiryagin, T. Godber, S. Hames, and M. Rasul (2015)
\newblock Conceptual designs of hybrid locomotives for application as heavy haul
trains on typical track lines,
\newblock \emph{Proceedings of the Institution of Mechanical Engineers Part F Journal of Rail and Rapid Transit}, vol. 227, no. 5,
pp. 439–452.

	
\bibitem[5]{p5}
F. Deng, X. Zhang, X. Li, T. Lei, and T. Wang (2018)
\newblock Decoupling control strategy for input-series output-parallel systems based on dual active bridge dc-dc converters
\newblock \emph{9th IEEE International Symposium on Power Electronics Distribution Generation System} , pp. 1–8.	
	
	
	\bibitem[6]{p6}
	Government Of India, Ministry of Railways (2021)
	\newblock Technical specification for traction alternators with auxiliary machines and their sub-assemblies used in broad gauge diesel electric ALCO locomotives
	\newblock \emph{Research, Design \& Standards Organization; Manak Nagar, Lucknow}.
	
\end{thebibliography}
\end{frame}

\begin{frame}
	\begin{columns}
		\column{2.5cm}
		\column{5cm}
		\Huge{~}
		\column{2.5cm}
	\end{columns}
\end{frame}
%\end{frame}
\end{document}